\newpage
\tbf{Abstract}: The abstract is a very brief summary of the report's contents. It should be about half page long. Somebody unfamiliar with your project should have a good idea of what it is about having read the abstract alone and will know whether it will be of interest to them.\\

\tbf{Acknowledgements}: It is usual to thank those individuals who have provided particularly useful assistance, technical or otherwise, during your project. Your supervisor will obviously be pleased to be acknowledged as they will have invested quite a lot of time overseeing your progress.\\

\tbf{Contents page}: This should list the main chapters and (sub) sections of your report. Choose self-explanatory chapter and section titles. If possible you should include page numbers indicating where each chapter/section begins. Try to avoid too many levels of subheading. Try if possible to stick to sections and subsections; subsubsections are usually avoidable.\\

\tbf{Introduction}: This is one of the most important components of the report. It should begin with a clear statement of what the project is about so that the nature and scope of the project can be understood by the reader. It should summarise everything you set out to achieve, provide a clear summary of the project's background and relevance to other work, and give pointers to the remaining sections of the report that contain the bulk of the technical material.\\

\tbf{Background and related work}: The background and related work section of the report should set the project into context by relating it to existing published work that you read at the start of the project when your approach and methods were being considered. There are usually many ways of approaching a given problem, and you should not just pick one at random. Describe and evaluate as many alternative approaches as possible. The published work may be in the form of research papers, articles, text books, technical manuals, or even existing software or hardware of which you have had hands-on experience. Do not be afraid to acknowledge the sources of your inspiration; you are expected to have seen and thought about other people's ideas, so your contribution largely will be putting them into practice in some other context.\\

\tbf{Body of report}: The central part of the report typically consists of three of four chapters detailing the technical work undertaken during the project. The structure of these chapters is highly project dependent. Usually they reflect the chronological development of the project, e.g., design, implementation, experimentation, and optimisation, although this is not always the best approach. However you choose to structure this part of the report, you should make it clear how you arrived at your chosen approach in preference to the other alternatives documented in the background. For implementation projects you should describe and justify the design of your system at some high level, for example by using any of the design methods taught during the first- and second-term courses, and should document any interesting problems with, or features of, your implementation. Integration and testing are also important to describe. Your supervisor will advise you on the most suitable structure for these middle sections.\\

\tbf{Evaluation:} All projects need to contain a serious and careful evaluation of their results. The specifics of the evaluation method (e.g., user study, experiments, formal proof review, etc.) are intrinsic to the nature of the project, so this is something that you must discuss and agree with your supervisor early in the project. Ideally, a presentation of the method and results of your evaluation should be included in its own separate section of the report.\\

\tbf{Conclusions and future work}: All good projects conclude with an objective evaluation of the project's successes and failures, and suggestions for future work that can take the project further. It is important to understand that there is no such thing as a perfect project. Even the very best pieces of work have their limitations and you are expected to provide a proper critical appraisal of what you have done. Your assessors are bound to spot the limitations of your work and you are expected to be able to do the same.\\

\tbf{Bibliography:} This consists of a list of all the books, articles, manuals, etc., used in the project and referred to in the report. You should provide enough information to allow the reader to find the source. You should give the full title and author, and you should state where it is published, including full issue number, date, and page numbers where necessary. In the case of a text book, you should quote the name of the publisher as well as the author(s).\\

\tbf{Appendices: } Appendices contain information that is peripheral to the main body of the report. Information typically included are things like program listings, tables, proofs, graphs, or any other material that would break up the theme of the text if it appeared in situ. Large program listings are rarely required, and should be compressed as much as possible, e.g., by printing in multiple columns and by using small font sizes, omitting inessential detail.\\

\tbf{User guide: } For projects that result in a new piece of software, you should provide a proper user guide providing easily understood instructions on how to install and use it. A particularly useful approach is to treat the user guide as a walk through of a typical session, or set of sessions, that collectively display all the features of your software. Technical details of how the software works is rarely required. Keep it concise and simple. The extensive use of diagrams illustrating the software in action prove particularly helpful. The user guide is sometimes included as a chapter in the main body of the report, but is often better as an appendix to the main report.\\